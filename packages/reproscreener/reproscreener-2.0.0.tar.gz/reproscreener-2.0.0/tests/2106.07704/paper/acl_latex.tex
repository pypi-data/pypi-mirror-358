\pdfoutput=1

\documentclass[11pt]{article}

\usepackage[]{acl}

\usepackage{times}
\usepackage{latexsym}

\usepackage[T1]{fontenc}

\usepackage[utf8]{inputenc}

\usepackage{microtype}


\usepackage[pdftex]{graphicx}
\usepackage{tabularx}
\usepackage{amsmath}
\usepackage{bm}
\usepackage{array}
\usepackage{enumitem}
\usepackage{wrapfig}
\usepackage{caption}
\usepackage{subcaption}
\usepackage{amsmath,amsfonts,bm}

\usepackage{booktabs}
\usepackage{multirow}
\usepackage[normalem]{ulem}
\useunder{\uline}{\ul}{}

\usepackage{algorithm}
\usepackage{algorithmic}
\renewcommand{\algorithmicrequire}{\textbf{Input:}}
\renewcommand{\algorithmicensure}{\textbf{Output:}}

\newcommand{\E}{\mathbb{E}}
\newcommand{\KLD}{\text{KL}}
\newcommand{\ENT}{\text{H}}
\newcommand{\BENT}{\mathbb{H}}
\newcommand{\BDist}{\mathbb{D}}
\newcommand{\MLE}{\text{MLE}}
\newcommand{\x}{\bm{x}}
\newcommand{\y}{\bm{y}}
\newcommand{\z}{\bm{z}}
\newcommand{\s}{\bm{s}}
\newcommand{\ba}{\bm{a}}
\newcommand{\bt}{\bm{t}}
\newcommand{\bc}{\bm{c}}
\newcommand{\yspace}{\mathcal{Y}}
\newcommand{\zspace}{\mathcal{Z}}
\newcommand{\tspace}{\mathcal{T}}
\newcommand{\btheta}{\bm{\theta}}
\newcommand{\bphi}{\bm{\phi}}
\newcommand{\bxi}{\bm{\xi}}
\newcommand{\bw}{\bm{\w}}
\newcommand{\dataset}{\mathcal{D}}
\newcommand{\loss}{\mathcal{L}}
\newcommand{\indicator}{\mathbb{I}}
\newcommand{\pdata}{p_{d}}
\newcommand{\pemp}{\tilde{p}_d}
\newcommand{\lm}{\text{LM}}









\title{
Efficient (Soft) $Q$-Learning \\ for Text Generation
with Limited Good Data
}


\author{
Han Guo$^1$,~~
Bowen Tan$^1$,~~
Zhengzhong Liu$^{1,2}$,~~
Eric P. Xing$^{1,2,4}$,~~
Zhiting Hu$^{3}$\\
$^1$Carnegie Mellon University,~~ $^2$Petuum Inc.,~~ $^3$UC San Diego,\\
$^4$Mohamed bin Zayed University of Artificial Intelligence\\
{\small 
{\tt \{hanguo, btan2, epxing\}@cs.cmu.edu, hectorzliu@gmail.com, zhh019@ucsd.edu}
}
}

\begin{document}
\maketitle
\begin{abstract}
Maximum likelihood estimation (MLE) is the predominant algorithm for training text generation models. This paradigm relies on direct supervision examples, which is not applicable to many emerging applications, such as generating adversarial attacks or generating prompts to control language models. Reinforcement learning (RL) on the other hand offers a more flexible solution by allowing users to plug in arbitrary task metrics as reward. Yet previous RL algorithms for text generation, such as policy gradient (\emph{on-policy} RL) and $Q$-learning (\emph{off-policy} RL), are often notoriously inefficient or unstable to train due to the large sequence space and the sparse reward received only at the end of sequences. In this paper, we introduce a new RL formulation for text generation from the soft $Q$-learning (SQL) perspective. It enables us to draw from the latest RL advances, such as path consistency learning, to combine the best of on-/off-policy updates, and learn effectively from sparse reward. We apply the approach to a wide range of novel text generation tasks, including learning from noisy/negative examples, adversarial attacks, and prompt generation. Experiments show our approach consistently outperforms both task-specialized algorithms and the previous RL methods.%
\footnote{Code at \url{https://github.com/HanGuo97/soft-Q-learning-for-text-generation}.}
\end{abstract}


\section{Introduction}






Off-policy reinforcement learning is an important research direction as the reuse of old experience promises to make these methods more sample efficient than their on-policy counterparts. This is an important property for many applications such as robotics where  interactions with the environment are very time- and cost-intensive.
Many successful off-policy methods make use of a learned Q-value function~\cite{td3,SAC,hessel2018rainbow,dqn15}. 
If the action space is discrete the Q-function can be directly  used to generate actions while for continuous action spaces it is usually used in an actor-critic setting where the policy is trained to choose actions that maximize the Q-function. In both cases accurate estimates of the Q-values are of crucial importance.

% \looseness=-1
% \looseness=-1 \spaceskip= 2pt plus 1pt minus 1.5pt  \spaceskip= 3pt plus 2pt minus 2pt
Unfortunately, learning the Q-function off-policy can lead to an overestimation bias~\cite{Thrun+Schwartz:1993}.
Especially when a nonlinear function approximator is used to model the Q-function, there are many potential sources of bias.
Different heuristics were proposed for their mitigation, such as the double estimator in the case of discrete action spaces~\cite{hasselt2016deepdouble} or taking the minimum of two estimates in the case of continuous actions~\cite{td3}.
While these methods successfully prevent extreme overestimation, due to their coarse nature, they can  still induce under- or overestimation bias to a varying degree depending on the environment~\cite{Lan2020Maxmin}.\looseness=-1

To overcome these problems we propose a principled and general method to alleviate the bias called Adaptively Calibrated Critics (ACC).
Our algorithm uses the most recent on-policy rollouts to determine the current bias of the Q-estimates and adjusts a bias controlling parameter accordingly.
This parameter adapts the size of the temporal difference (TD) targets  such that the bias can be corrected in the subsequent updates.
As the parameter changes slower than the rollout returns, our method still benefits from stable and low-variance temporal difference targets, while it incorporates the information from unbiased but high variance samples from the recent policy to reduce the bias. 


{\spaceskip= 2pt plus 1pt minus 1.5pt  \spaceskip= 3pt plus 2pt minus 2pt We apply ACC to Truncated Quantile Critics (TQC) \cite{tqc}, which is a recent off-policy actor-critic algorithm for continuous control showing strong performance on various tasks. 
In TQC the bias can be controlled in a finegrained way with the help of a hyperparameter that has to be tuned for every environment.
ACC allows to automatically adjusts this parameter online during the training in the environment.
As a result, it eliminates the need to tune this hyperparameter in a new environment, which is very expensive or even infeasible for many applications.}

% \looseness=-1
We evaluate our algorithm on a range of continuous control tasks from OpenAI gym \cite{gymopenai} and robotic tasks from the meta world benchmark \cite{yu2020meta} and exceed the current state-of-the-art results among all algorithms that do not need  tuning of environment-specific hyperparameters.
For each environment, ACC matches the performance of TQC with the optimal hyperparameter for that environment.
Further, we show that the automatic bias correction allows to increase the number of value function updates performed per environment step, which results in even larger performance gains in the sample-efficient regime.
We additionally apply ACC to the TD3 algorithm \cite{td3} where it also leads to notably improved performance, underscoring the generality of our proposed method.
% \linepenalty 
To summarize, the main contributions of this work are:
\begin{enumerate}[leftmargin=0.72cm]
    \item We propose Adaptively Calibrated Critics, a new general algorithm  that reduces the bias of value estimates in a principled fashion with the help of the most recent unbiased on-policy rollouts.
    \item As a practical implementation we describe how ACC can be applied to learn a bias-controlling hyperparameter of the TQC algorithm and show that the resulting algorithm sets a new state of the art on the OpenAI continuous control benchmark suite.
    \item ACC achieves strong performance on robotics tasks.
    % \item We demonstrate that ACC is a general algorithm by additionally applying it successfully to TD3.
    \item We demonstrate that ACC is a general algorithm with respect to the adjusted parameter by additionally applying it successfully to TD3.
    % \item We evaluate the resulting algorithm and show that it sets a new state of the art  on the OpenAI continuous control benchmark suite.
\end{enumerate}
\looseness=-1

% To summarize the main contributions of this work, we show:
% \begin{enumerate}[leftmargin=0.72cm]
%     \item We propose Adaptively Calibrated Critics, a new general algorithm  that reduces the bias of value estimates in a principled fashion with the help of the most recent unbiased on-policy rollouts.
%     \item As a practical implementation we describe how ACC can be applied to learn a bias-controlling hyperparameter of the TQC algorithm and show that the resulting algorithm sets a new state of the art on the OpenAI continuous control benchmark suite.
%     \item We show that ACC achievs strong performance on robotics tasks.
%     % \item We demonstrate that ACC is a general algorithm by additionally applying it successfully to TD3.
%     \item We demonstrate that ACC is a general algorithm with respect to the adjusted parameter by additionally applying it successfully to TD3.
%     % \item We evaluate the resulting algorithm and show that it sets a new state of the art  on the OpenAI continuous control benchmark suite.
% \end{enumerate}

To allow for reproducibility of our results we describe our algorithm in detail, report all hyperparameters, use a large number of random seeds for evaluation, and made the source code publicly available\footnote{\url{https://github.com/Nicolinho/ACC}}. 







\section{Background and Challenges}
\label{subsec:preliminaries}

We aim to learn a generation model
$p_{\theta} (\y) = \prod_{t=0}^T p_{\theta}(y_t | \y_{<t})$, where
$y_t$ is a token from a vocabulary $\mathcal{V}$. The distribution at each step $t$ is obtained by applying softmax on the logits $f_\theta(y | \y_{<t})$:
\begin{equation}
\small
\begin{split}
    p_\theta(y_t | \y_{<t}) = \frac{\exp f_\theta(y_t | \y_{<t}) }{ \sum_{y'\in\mathcal{V}} \exp f_\theta(y' | \y_{<t} ) }.
\end{split}
\label{eq:softmax-logit}
\end{equation}
Despite its popularity, MLE-based training only applies when clean supervised data $\bm{y}^*$ is available, and cannot be used to optimize arbitrary task metrics (e.g., BLEU, entailment score) which are typically the goal in many text generation tasks. 
\label{sec:background:rl}
Previous research has formulated text generation as an RL problem by considering the following finite-time Markov Decision Process (MDP). 
At each time step $t$, let the ``state'' be $\bm{s}_t = \bm{y}_{< t}$, namely the partial sequence generated so far. The model (``agent'') takes as input the current state $\bm{s}_t$ and outputs a token (``action'') $a_t \in \mathcal{V}$ according to a policy $\pi(a_t | \bm{s}_t)$. 
The agent then receives a reward $r_t = r(\bm{s}_t, a_t)$ and deterministically transitions to next state $\bm{s}_{t+1}$ (i.e., the concatenation of the tokens in $\s_t$ and the new token $a_t$). 

Following the notation convention in RL, let $\tau$ be the trajectory (i.e., text sample) generated by the policy. The agent's objective is to maximize the accumulative reward,
$J(\pi) = \mathbb{E}_{\tau \sim \pi}\left[ \sum^T_{t=0} \gamma^t r_t \right]$,
where $\gamma$ is the discount factor.
A central concept in RL is the $Q$-function of policy $\pi$, $Q^\pi(\s_t, a_t)=\E_{\pi}\left[ \sum_{t'=t}^T \gamma^{t'} r_{t'} \mid \s_t, a_t \right]$, the expected future reward of taking action $a_t$ (i.e., generating token $a_t$) in state $\s_t$ and continuing with the policy $\pi$. 

\noindent\textbf{Challenges. } Text generation poses significant challenges to RL, particularly because (1) the reward signal is usually sparse, i.e., $r_t = 0,\ \forall t<T$ and the agent receives a non-zero reward $r_{T}$ only after it generates the full sequence, (2) the action space (i.e., the vocabulary $\mathcal{V}$) is extremely large.
The challenges have led to difficulties of the two major families of RL approaches applied to text generation problems, as detailed below.

\noindent\textbf{Policy-based RL}
techniques directly parameterize the policy $\pi_{\theta}$ with parameters $\btheta$. Thus the policy $\pi_{\theta}(a_t | \s_t)$ exactly corresponds to the above generation model $p_{\theta}(y_t | \y_{<t})$. 
\emph{Policy gradient (PG)} is one of the most widely used algorithms for text generation \citep{ranzato2015sequence}.
It optimizes the cumulative reward with the policy gradient using the estimated $Q^{\pi_\theta}$ value based on sample $\tau$. PG is an \textit{on-policy} algorithm, meaning that the sample $\tau$ needs to come from the the current policy $\pi_\theta$ itself.
In practice, however, optimizing this objective alone from scratch is unlikely going to work because most samples $\tau\sim\pi_\theta$ are just gibberish with zero reward, failing to provide meaningful training signals for updating the policy. 
Previous literature either initializes the policy $\pi_{{\theta}}$ with MLE training, and/or use a combination of MLE and PG updates, which often leads to marginal gains in practice~\citep{wu2018study,Choshen2020On}. 



\noindent\textbf{Value-based RL}
techniques, such as \emph{$Q$-learning}, implicitly learn the policy $\pi$ by approximating the value $Q^{\pi}(\bm{s}, a)$ directly.
Deep $Q$-learning~\citep{mnih2013playing} parameterizes the $Q$-function as $Q_\theta(\x,a)$, and train the parameters by minimizing the following regression objective $\loss(\btheta)$ based on the Bellman temporal consistency:
\begin{equation}
\small
    {\E}_{\pi'}\left[ \frac{1}{2} \left( r_t {+} \gamma \max_{a_{t+1}} Q_{\bar{\theta}}(\s_{t+1}, a_{t+1}) {-} Q_{\theta}(\s_t, a_t) \right)^2 \right]%
\label{eq:q-regression}
\end{equation}
where $\bar{\btheta}$ is the parameters of the \emph{target} $Q$-network, which is a slow copy of $\btheta$ and considered as constant for gradient computation of $\btheta$. 
Here $\pi'$ is an \emph{behavior policy} which can be an arbitrary distribution over text, such as the data distribution or replay buffer \citep{mnih2013playing}. 
This makes $Q$-learning an \textit{off-policy} algorithm because of its ability to use samples coming from other policies. After learning $Q_\theta$, one can induce a policy $\pi$ from it that takes $\arg\max_{a} Q_\theta(\s, a)$ at each state $\s$.
\citet{jaques2017sequence} instead sample tokens from the softmax function applied to $Q_\theta$.

However, the training can be unstable and inefficient due to several challenges: 
{\bf (1)} The bootstrapping nature of the above regression problem can make the training unstable. That is, the regression target $r_t + \gamma \max_{a_{t+1}} Q_{\bar{\theta}}(\s_{t+1}, a_{t+1})$ itself is derived from the $Q$-function to be learned \citep{kumar2019stabilizing}. The problem is exacerbated in the presence of sparse reward in text generation, where the real observed signal $r_t$ is zero for all intermediate $t<T$; 
{\bf (2)} The large action space (e.g., $10^4$) in text generation results in slow updates. In particular, notice that Eq.\eqref{eq:q-regression} applies the gradient update to the $Q_{{\theta}}$-value of the \emph{only one} particular token $a_t$ (out of, say, the $10^4$ candidate tokens in the vocabulary), making the training inefficient;
{\bf (3)} Besides, pure off-policy updates could be highly sensitive to the quality of training data, and miss the opportunity of on-policy exploration that maximizes the reward of interest in a more direct way.



\begin{figure*}
    \centering
    \includegraphics[width=0.99\linewidth]{figures/soft-q-learning.pdf}
    \vspace{-4pt}
    \caption{
    Soft $Q$-Learning with path consistency learning (PCL) objectives.
    {\bf Left:} Single-step objective (Eq.\ref{eq:pcl-loss}), where for each $(\s_t, a_t)$, the computation involves step $t$ and $t+1$. Dashed boxes in \textcolor[HTML]{3A6B73}{\textbf{dark green}} and \textcolor[HTML]{808285}{\textbf{gray}} indicate the regression target, where the intermediate reward $r_t$ is often 0 due to sparsity. The gradient is applied to parameters $\btheta$ at step $t$ (indicated by {\textcolor{orange}{\textbf{orange}}} color). {\bf Right:} Multi-step objective (Eq.\ref{eq:multi-loss}) which aggregates from step $t$ all the way to $T$. In this way, the final-step non-zero reward $r_T$ is used as the regression target.
    }
    \vspace{-4pt}
    \label{fig:soft-q-learning-pcl}
\end{figure*}



\section{The Soft $Q$-Learning Framework}


We introduce the soft $Q$-learning (SQL) formulation of text generation. It is seamlessly compatible with the common architecture of text generation model (Eq.\ref{eq:softmax-logit}), permits easy implementation (\S\ref{sec:sql-formulation}),
and enables efficient and stable RL training in practice (\S\ref{subsec:method:pcl}).
Figure~\ref{fig:soft-q-learning-pcl} and Algorithm~\ref{alg:sql} summarizes the resulting SQL framework.




\subsection{SQL Formulation for Text Generation}\label{sec:sql-formulation}

Soft $Q$-learning \citep{haarnoja2017reinforcement,schulman2017equivalence,nachum2017bridging} is an maximum-entropy (MaxEnt) extension to the standard (hard) $Q$-learning~\citep{mnih2015human,sutton2018reinforcement}. Under this framework, the agent is encouraged to optimize the reward while staying as stochastic as possible, with the objective
$J_{\text{MaxEnt}}(\pi)=\mathbb{E}_{\tau \sim \pi}\left[\sum_{t=0}^T \gamma^t r_t + \alpha \mathcal{H}\left(\pi\left(\cdot \mid \bm{s}_{t}\right)\right)\right]$,
which augments the vanilla $J(\pi)$
with the additional Shannon entropy term $\mathcal{H}$ with coefficient $\alpha$.\footnote{WLOG, we can assume $\alpha{=}1$, as it can be folded into the reward function by scaling the latter with $1/\alpha$.}
This is appealing because it seamlessly connects the $Q$-values to the familiar output \textit{logits} of a text generation model, which enables straightforward implementation of the SQL formulation.

\noindent\textbf{$Q$-values as Generation Model Logits.}
We show the connection of the $Q$-values with the logits, i.e., outputs right before the $\operatorname{softmax}$ layer.
Concretely, with the SQL objective,
the following relationship between optimal policy $\pi^*$ and action-value $Q^*$ holds~\citep{haarnoja2017reinforcement,schulman2017equivalence}:
\begin{equation}
\small
    \pi^{*}(a | \bm{s})=\frac{\exp Q^{*}(\bm{s}, a)}{\sum_{a^{\prime}} \exp Q^{*}\left(\bm{s}, a^{\prime}\right)}.
    \label{eq:optimal-pi-and-q}
\end{equation}
This form is highly reminiscent of the $\operatorname{softmax}$ layer of the generation model in Eq.\eqref{eq:softmax-logit}. The connection suggests that we can naturally parameterize the $Q$-function in SQL as the generation model logit function, i.e., $Q_\theta(\s, a) \equiv f_\theta(a|\s)$. In other words, \emph{the model output $f_\theta(a|\s)$, originally interpretted as the ``logit'' of token $a$ given the preceding tokens $\s$, is now re-interpretted as the $Q$-value of action $a$ in state $\s$.} When achieving optimality, $f_{\theta^*}(a|\s)$, namely $Q^{*}(\bm{s}, a)$, represents the best possible future reward achievable by generating token $a$ in state $\bm{s}$. Similarly, the full generation model $p_\theta(a|\s)$ in Eq.\eqref{eq:softmax-logit} that applies $\operatorname{softmax}$ to $f_\theta$ now precisely corresponds to the policy $\pi_\theta$ induced from $Q_\theta(\s, a)$. That is,
\begin{equation}
\small
\begin{aligned}
    \pi_\theta(a | \bm{s})
    &= \frac{\exp Q_\theta(\bm{s}, a)}{\sum_{a^{\prime}} \exp Q_\theta\left(\bm{s}, a^{\prime}\right)} \\
    &\equiv \frac{\exp f_\theta(a | \bm{s})}{\sum_{a^{\prime}} \exp f_\theta\left(a^{\prime} | \bm{s} \right)} 
    = p_\theta(a | \s).
    \label{eq:pi-and-q-theta}
\end{aligned}
\end{equation}
We could further gain even more intuitive interpretation of the above generation policy $\pi^*$ from the lens of \emph{advantage} function  \citep{sutton2018reinforcement}. Specifically, in SQL, the optimal \emph{state-value} function is the log-normalizer of the optimal $Q$-values \citep{haarnoja2017reinforcement,schulman2017equivalence}.
This allows a more concise form of Eq.\eqref{eq:optimal-pi-and-q}:
\begin{equation}
\small
\begin{aligned}
    V^{*}\left(\bm{s}\right) &= \log \sum\nolimits_{a'} \exp Q^{*}\left(\bm{s}, a'\right) \\ %
    \pi^{*}(a | \bm{s}) &= \exp \big(Q^*(\bm{s}, a) - V^*(\bm{s})\big) = \exp A^*(\bm{s}, a),
    \label{eq:sql-state-value}
\end{aligned}
\end{equation}
where $A^*$ is the optimal advantage function. The equation says that, in the proposed text generation SQL formulation, the optimal policy generates token $a$ in state $\s$ according to the token's advantage.







\subsection{Efficient Training with Path Consistency}
\label{subsec:method:pcl}

Vanilla training based on the Bellman temporal consistency can suffer from the instability and inefficiency issues similar to the conventional $Q$-learning (\S\ref{sec:background:rl}), as we discuss more in the appendix (\S\ref{appendix-subsubsec:vanilla-training-with-temporal-consistency}). Fortunately, our SQL formulation allows us to import latest advances of RL techniques to overcome the difficulties.
Specifically, we adapt the \emph{unified path consistency learning (PCL)} that has excelled in game control~\citep{nachum2017bridging}.
The PCL-based training updates $Q$-values of \emph{all} tokens at once through a connection between the value function and the induced policy.
More specifically,~\citet{nachum2017bridging} showed that the optimal policy $\pi^*$ (Eq.\ref{eq:optimal-pi-and-q}) and the optimal state value function $V^*$ (Eq.\ref{eq:sql-state-value}) in SQL must satisfy the following consistency property for 
all states and actions:
\begin{equation}
\small
    V^*\left(\bm{s}_{t}\right)-\gamma V^*\left(\bm{s}_{t+1}\right) = r_t - \log \pi^*\left(a_{t} | \bm{s}_{t}\right), \forall \bm{s}_t,\  a_t.
    \label{eq:pcl}
\end{equation}
Accordingly, the PCL-based training attempts to encourage the satisfaction of the consistency with the following regression objective $\loss_{\text{SQL, PCL}} (\bm{\theta})$:
\begin{equation}
\small
\begin{aligned}
    {\mathbb{E}}_{\pi'}\left[\frac{1}{2}\bigg({-}V_{\bar{{\theta}}}\left(\bm{s}_{t}\right){+}\gamma V_{\bar{{\theta}}}\left(\bm{s}_{t+1}\right) {+} 
    r_t {-} \log \pi_{{\theta}}\left(a_{t} | \bm{s}_{t}\right)\bigg)^{2}\right],
    \label{eq:pcl-loss}
\end{aligned}
\end{equation}
where $\pi_\theta$ is the induced policy defined in Eq.\eqref{eq:pi-and-q-theta}; $V_{\bar{{\theta}}}$ is defined similarly as in Eq.\eqref{eq:sql-state-value} but depends on the target $Q_{\bar{\theta}}$ network (i.e., a slow copy of the $Q_\theta$ to be learned), and recall that $\pi'$ is an arbitrary behavior policy (e.g., data distribution). 
Please see Figure~\ref{fig:soft-q-learning-pcl} (left) for an illustration. 
Crucially, notice that the gradient update is applied to $\btheta$ through the $\log \pi_\theta$ term which explicitly involves the $Q_\theta$-values of \emph{all} tokens $a$ in the vocabulary. This shows an important difference from the above vanilla training in conventional $Q$-learning (\S\ref{sec:background:rl}) where $Q_\theta$ is updated only through the particular $a_t$ token. The PCL training thus offers more efficient updates for the $Q_\theta$ function.
In the appendix (\S\ref{comparison-with-mle-objective}), we also discuss the difference from the MLE objective. Intuitively, MLE trains the model to (blindly) increase the probability of the observed tokens, while PCL encourages the (log) probability of the tokens to match the approximate advantage values.



\paragraph{Multi-step PCL for Sparse Reward.} 
The above PCL objective Eq.\eqref{eq:pcl-loss} alone does not resolve the potential instability issue due to the bootstrapped $V_{\bar{{\theta}}}(\s_{t+1})$ value and the sparse reward (i.e., $r(\s_t, a_t)=0$ for $t<T$). 
Our SQL formulation allows us to additionally incorporate the \emph{multi-step} variant of the PCL training \citep{nachum2017bridging} to resolve the issue. Specifically, by applying a telescoping sum on the consistency equation (Eq.\ref{eq:pcl}) starting from $t$ up to $T$, we arrive at the multi-step temporal consistency:
\begin{equation}
\small
\begin{aligned}
    &V^{*}\left(\bm{s}_{t}\right){-}\gamma^{T-t} V^{*}\left(\bm{s}_{T+1}\right)
    {=} \sum_{l=t}^{T} \gamma^{l-t}\big(r_{l}{-}\log \pi^{*}\left(a_{l} | \bm{s}_{l}\right)\big),
\end{aligned}
\end{equation}
where the value of past-terminal state is zero, $V^*\left(\bm{s}_{T+1}\right) = 0$; and the rewards are only available at the end, $\sum_{l=t}^{T} \gamma^{l-t} r_{l} = \gamma^{T-t} r_{T}$. We can then come to the following multi-step objective function $\loss_{\textrm{SQL, PCL-ms}}(\btheta)$, 
\begin{equation}
\small
    \mathbb{E}_{\pi'} \left[\frac{1}{2}\left({-}V_{\bar{{\theta}}}\left(\bm{s}_{t}\right){+}\gamma^{T-t} r_{T}{-}\sum_{l=t}^{T} \gamma^{l-t} \log \pi_{{\theta}}\left(a_{l} | \bm{s}_{l}\right)\right)^{2}\right].
    \label{eq:multi-loss}
\end{equation}
We can see the objective side-steps the need to bootstrap intermediate value functions $V_{\bar{{\theta}}}(\bm{s}_{t'})$ for $t' > t$. Instead, it directly uses the non-zero end reward $r_T$ to derive the update for $\btheta$. Please see Figure~\ref{fig:soft-q-learning-pcl} (right) for an illustration. 
In practice, we combine the single- and multi-step objectives (Eqs.\ref{eq:pcl-loss} and \ref{eq:multi-loss}) together for training.

\paragraph{Joint On- and Off-policy Training.} 
Finally, we highlight that the behavior policy $\pi'$ involved in the objectives Eqs.\eqref{eq:pcl-loss} and \eqref{eq:multi-loss} can be an arbitrary policy.
For example, $\pi'$ can be a (possibly noisy) text dataset, or a set of text samples produced by other generation models, resulting in off-policy training. We can also set $\pi'$ to be the current generation model $\pi_\theta$ to be learned, resulting in on-policy training. In practice, we could first train the model with only off-policy data for warming up, and then continue with joint on- and off-policy training to further maximize the reward.

\begin{algorithm*}[!h]
\caption{Efficient Soft $Q$-Learning for Text Generation}
\label{alg:sql}
\begin{algorithmic}[1]
\REQUIRE $Q_\theta$ (i.e., generation model logit function $f_\theta$ in Eq.\ref{eq:softmax-logit}) \\
\quad\ \  Reward function $r(\s, t)$ \\
\quad\ \  Training examples $\mathcal{D}$ (for off-policy updates; {\it optional}) \\
\STATE Initialize ${\bm{\theta}}$ and target model parameters ${\bar{\bm{\theta}}}$
\REPEAT
    \STATE Draw a batch of off-policy samples $\{\tau_{\text{off}}\} \sim \mathcal{D}$
    \STATE Draw a batch of on-policy samples $\{\tau_{\text{on}}\}$ by decoding with policy $\pi_\theta(a_t\mid\s_t)$ (Eq.\ref{eq:pi-and-q-theta})
    \STATE Compute $Q_{\theta} (\bm{s}_t, a_t)$ values (the model logits) and target $Q_{\bar{\theta}} (\bm{s}_t, a_t)$ for $(\bm{s}_t, a_t) \in \{\tau_{\text{off}}\} \cup \{\tau_{\text{on}}\}$
    \STATE Compute the objectives in Eqs.\eqref{eq:pcl-loss} and \eqref{eq:multi-loss}
    \STATE Update the model parameters $\bm{\theta}$ via gradient descent
    \STATE Update the target model parameters $\bar{\bm{\theta}}$ by $\bar{\bm{\theta}} \leftarrow \rho \bar{\bm{\theta}} + ( 1 - \rho) \bm{\theta}$ with update rate $\rho$
\UNTIL{convergence}
\ENSURE The trained $Q_{\theta^*}$ and the induced generator $\pi_{\theta^*}$
\end{algorithmic}
\end{algorithm*}



\section{Applications and Experiments}


We show broad applications of the proposed RL text generation framework to a variety of problems where no clean supervision data is available. These include learning with noisy or even negative data (\S\ref{subsec:noisy-data}), generating adversarial text attacks (\S\ref{subsec:adversarial-attack}), and generating prompts to steer pretrained LMs (\S\ref{subsec:prompt-generation}).
We also study the performance on standard supervised generation tasks (\S\ref{appendix-subsec:standard-tasks}) and show that our approach is competitive to train text generation models \emph{from scratch}. We provide detailed configurations in the appendix (\S\ref{appendix-subsec:setup-details}).

\subsection{Learning from Noisy (Negative) Text}
\label{subsec:noisy-data}

The popular MLE algorithm learns by (blindly) imitating training data.
However,
it is often expensive to curate clean quality data. It is thus highly desirable to be able to learn from data with noises, or even \emph{negative} examples.
With the guidance of task metrics (rewards), the model can even learn to ``outperform'' the training data and achieve desired generation behaviors.
To this end, we consider the task of \emph{entailment generation}~\citep{pasunuru2017multi}. Given a sentence (premise), the goal is to generate a new sentence (hypothesis) that logically follows the premise. 



\begin{figure*}
    \centering
    \includegraphics[width=0.99\linewidth]{figures/202101005-entailment-combined.pdf}
    \vspace{-7pt}
    \caption{
    \textbf{Left:} entailment generation performance plotted against diversity (average of $H_1$ and $H_2$). 
    Circles represent results of top-$p$ sample outputs,
    and triangles represent results of beam-search outputs.
    Please see Table~\ref{table:entailment-generation} for additional results.
    \textbf{Right:} entailment attack performance against diversity. 
    Only a few \texttt{MLE+PG} dots are visible because the model is not able to generate more diverse samples even with increasing $p$ value in top-$p$ decoding, i.e., the model collapses.
    }
    \label{fig:entailment-generation}
    \label{fig:entailment-attack}
    \vspace{-5pt}
\end{figure*}


\paragraph{Setup (more in~\S\ref{appendix-subsubsec:setup-noisy-data}).}
We sub-sampled $50k$ training examples from the SNLI dataset~\citep{bowman2015large}, a commonly used entailment classification dataset. The hypotheses have an average entailment probability of only $50\%$, and over $2/5$ of them less than $20\%$ (negative/contradictive examples) -- a significant challenge for the models to learn from the noises. 
The rewards include (1) the entailment score of the generation measured by a robust entailment classifier~\citep{nie2020adversarial}, (2) the log-likelihood of the generation as an indicator of language quality measured by a GPT-2 language model~\citep{radford2019language}, and (3) BLEU score w.r.t the input premises as another language quality reward that avoids trivial outputs. We sum together all rewards with weights $1.0$.

We compare our approach with a broad range of baselines, including 
(1) the standard MLE training (\texttt{MLE}); 
(2) \texttt{MLE+reward}, where we use the reward function to filter examples;
(3) joint MLE and PG training with MLE initialization (\texttt{MLE+PG}), where we initialize the model with MLE training, then train it with combined MLE and PG losses;
previous text-generation RL algorithms including (4) \texttt{MIXER}~\citep{ranzato2015sequence}, 
(5) \texttt{Self-critic}~\citep{rennie2017self}, 
and 
(6) one of the latest methods \texttt{GOLD-$s$}~\citep{pang2021text} which is a pure off-policy method based on importance-sampling PG. 
To ablate the effect of multi-step training (\S\ref{subsec:method:pcl}), we additionally compare with a simplified variant of our approach that uses only vanilla single-step PCL training (\texttt{SQL(single)}). We include more baselines such as MLE weighted by rewards in \S\ref{appendix-subsubsec:experiments-noisy-data}.




We evaluate generation results in terms of entailment rate, language quality (perplexity), and diversity which is measured by the Shannon entropy over unigrams and bigrams ($H_1$, $H_2$) \citep{gehrmann2021gem}. Since text generation models intrinsically
trade off diversity and quality \citep{caccia2019language,hashimoto2019unifying}, we vary the generation diversity by generating samples via top-$p$ sampling \citep{holtzman2019curious} with different $p$ values, and plot the entailment rate and perplexity against diversity, resp. We also evaluate the samples produced by beam-search decoding.

\paragraph{Results.}
Figure~\ref{fig:entailment-generation} (left) shows the results, and Table~\ref{table:entailment-generation-examples-sql} shows samples.
First, notice that \texttt{MLE} performs poorly, while \texttt{MLE+reward} improves upon it. This is not surprising as the training data contain noisy/negative examples. Similarly, since the pure off-policy algorithm \texttt{GOLD-$s$} relies heavily on the data distribution, we observed that it achieves sub-optimal performance. The on-policy \texttt{MLE+PG} with MLE initialization gives better entailment rate.
In comparison, our full \texttt{SQL} framework achieves the best entailment-diversity trade-off. 
The comparison between \texttt{SQL} and \texttt{SQL(single)} highlights the importance of having the multi-step objective which directly uses the end reward rather than bootstrapping intermediate $Q$-values for supervision.



















\subsection{\textit{Universal} Adversarial Attacks}
\label{subsec:adversarial-attack}

We next study the application
in text adversarial attacks, where again no supervised data is available. 
Adversarial attacks is an increasingly important research topic as they reveal models' vulnerabilities and flaws. This is especially true for universal attacks~\citep{wallace2019universal,atanasova2020generating}, where we want to generate universal examples that trick the model on \emph{all} possible inputs. 
For instance, consider the context of entailment classification.
Our goal is to find universal human-readable hypotheses that are going to be classified as ``entailment'' with as high probability as possible, regardless of the input premises. This is a more challenging setting compared to previous instance-specific attack \citep{morris2020textattack,jin2020bert,ebrahimi2017hotflip} where the attack model conditions on a premise and generates an adversarial hypothesis specific to the premise. 


\paragraph{Setup (more in~\S\ref{appendix-subsubsec:setup-adversarial-attack}).}
We aim to attack one of the most popular MultiNLI~\citep{williams2018broad} entailment classifiers on HuggingFaceHub.\footnote{\url{https://github.com/pytorch/fairseq/tree/master/examples/roberta}
} The attack generation model generates adversarial text without conditioning on any inputs so that the generated attacks are universal to all premises.  
We compare our \texttt{SQL} with \texttt{MLE+PG}. We use all hypotheses in the MultiNLI dataset as the training data for the MLE training in \texttt{MLE+PG} and the off-policy updates for our \texttt{SQL}.
We do not compare with previous specialized adversarial text attack methods, because they either are not applicable to the challenging universal attack setting \citep{morris2020textattack,jin2020bert,ebrahimi2017hotflip}, or were not designed to generate human-readable sentences \citep{wallace2019universal}. 
We use similar settings as in \S\ref{subsec:noisy-data} to explore the diversity-quality trade-off 
by plotting the entailment rate and perplexity against diversity, respectively. 
The entailment classifier to be attacked is used as entailment score reward functions. We also include a token-level repetition penalty reward for readability.


\paragraph{Results.}

Figure~\ref{fig:entailment-attack} (right) shows the results, and Table~\ref{table:entailment-attack-examples} shows samples.
We can see that \texttt{SQL} outperforms \texttt{MLE+PG} consistently across different diversity values. The outputs from \texttt{MLE+PG} are not diverse even with high $p$'s, indicating the model collapses and can only generate a small set of unique adversarial examples. 
The model by \texttt{SQL} discovers the pattern ``saint-pierre-et-saint-paul'' (an entity name), and exploits this to generate samples with high universal entailment rate.


\begin{figure}[t]
    \centering
    \begin{minipage}{.49\textwidth}
        \centering
        \includegraphics[width=0.75\linewidth]{figures/20210527.prompt-generation.pdf}
        \vspace{-7pt}
        \caption{Average topic accuracy. Please see Table~\ref{table:prompt-generation-full} for more details.}
        \label{fig:prompt-generation-topic}
\end{minipage}%
\hfill
\begin{minipage}{0.49\textwidth}
\centering
\small
\begin{tabular}{@{}llll@{}}
\toprule
\textbf{PPLM} & \textbf{GeDi} & \textbf{MLE (5)} & \textbf{SQL (off, 5)} \\
\midrule
$13.07$ & $123.88$ & $25.70$ & $25.77$ \\
\midrule
\multicolumn{2}{l}{\textbf{MLE+PG (5/10/15)}} & \multicolumn{2}{l}{\textbf{SQL (5/10/15, ours)}} \\
\midrule
\multicolumn{2}{l}{$25.52$/$28.16$/$28.71$} & \multicolumn{2}{l}{$25.94$/$26.95$/$29.10$} \\
\bottomrule
\end{tabular}
\vspace{-7pt}
\captionof{table}{
Average perplexity across topics. The lower, the more fluent the generated continuation sentences.
}
\vspace{-5pt}
\label{table:prompt-generation}

\vspace{9pt}

\centering
\small
\begin{tabular}{p{0.79cm}p{0.6cm}p{0.6cm}p{0.6cm}}
\toprule
\textbf{Model} & PPLM & GeDi & SQL \\
\midrule
\textbf{Seconds} & $5.58$ & $1.05$ & $0.07$ \\
\bottomrule
\end{tabular}
\vspace{-7pt}
\caption{Average sentence generation time cost.}
\vspace{-7pt}
\label{table:prompt-generation-speed}
\end{minipage}
\vspace{-7pt}
\end{figure}


\subsection{Prompt Generation for Controlling Pretrained Language Models}
\label{subsec:prompt-generation}





A reward function does not just have to be a metric like the BLEU score, but also a complicated pipeline that eventually returns a score.
To demonstrate this, we consider the emerging task of prompting a large pretrained LM for controllable generation \citep{Hu2017TowardCG,radford2019language,NEURIPS2020_1457c0d6}.
The goal is to learn to generate text prompts that steer the LM to generate sentences of certain desired attributes (e.g., topics). 
The problem of controlling the generation of pretrained LMs was previously approached through specialized algorithms such as modifying the LM hidden states during decoding~\citep{Dathathri2020Plug,krause2020gedi,qin2020backpropagation}. Here we show that prompts offer an easier, faster, more effective way for controlled generation.

Learning to generate/tune prompts is gaining increasing attention recently.
It side-steps the needs for expensive LM fine-tuning, and adapts LMs to new scenarios with prompt as the (compute-friendly) interface.
Most existing approaches~\citep{wallace2019universal,li2021prefix,lester2021power} rely on gradient backpropagation and are applicable only when the whole training pipeline is differentiable.
This does not hold for the text generation setting, as illustrated in Figure~\ref{fig:prompt-generation-flow}. In contrast, the RL framework is generally applicable to any differentiable or discrete pipelines.


\begin{figure}
\centering
\includegraphics[width=0.99\linewidth]{figures/prompt-generation-task-new.pdf}
\vspace{-12pt}
\caption{
The scheme of prompt generation for controlling the outputs of pretraind LMs.
}
\vspace{-7pt}
\label{fig:prompt-generation-flow}
\end{figure}


\paragraph{Setup (more in~\S\ref{appendix-subsubsec:setup-prompt-generation}).}
Following \citep{dathathri2019plug}, we aim to control the generation to have one of 7 topics (e.g., ``science''); the generated prompt is prepended to one of 20 input sentences for the pretrained LM to generate continuation sentences. Figure~\ref{fig:prompt-generation-flow} shows the architecture of prompt-based controllable generation. We compare our \texttt{SQL} method with \texttt{MLE+PG} as before. Since the prompt length could impact the generated sentences, we conducted experiments with maximum prompt length $5$, $10$, and $15$. As ablation study, we also evaluate the SQL algorithm with only off-policy updates (i.e., without on-policy exploration), denoted as \texttt{SQL(off)}, and compare it with vanilla \texttt{MLE} training. Finally, we also compare with two specialized controllable generation techniques based on pretrained LMs, namely \texttt{PPLM}~\citep{dathathri2019plug} and \texttt{GeDi}~\citep{krause2020gedi}, following similar procedures using their open-sourced code. We use a distilled GPT-2 model\footnote{\url{https://huggingface.co/distilgpt2}} as the pretrained LM to be controlled. %
For rewards, we use the topic accuracy of the continuation sentences measured by a \emph{zero-shot} classifier, plus the the log-likelihood of continuation sentences as the language quality reward measured by a distilled GPT-2.\footnote{Note that the language quality emphasis is on the generated sentences. Prompts themselves do not necessarily have to be human-readable~\cite{wallace2019universal,sheng2020towards}.}


\paragraph{Results.}

Figure~\ref{fig:prompt-generation-topic} shows the topic accuracy of the controlled LM outputs averaged across the 7 topics, and Table~\ref{table:prompt-generation} shows the respective language quality results. More detailed topic accuracy results and samples are provided in the appendix (\S\ref{appendix-subsubsec:experiments-prompt-generation}) (where \texttt{GeDi} obtained low accuracy on 2 of the 7 topics, possibly because the topic tokens are tokenized into two subwords for which the model released by the authors was not specifically trained). 
We can see that the prompts generated by our \texttt{SQL} cause the LM to generate sentences with high topic accuracy while maintaining low perplexity in most settings.
Increasing the prompt length positively impacts the topic accuracy, which makes sense because longer prompts give more flexible for steering the LM.
The comparison between \texttt{MLE} and \texttt{SQL(off)}
shows that the off-policy component of SQL is better than standard MLE training, as it incorporates reward signals instead of just blindly following the (noisy) data. 

Next, comparing with the previous steered decoding such as \texttt{PPLM} and \texttt{GeDi},
we can see the prompt-based control trained with RL achieves better trade-off between topic accuracy and language quality. Moreover, once a prompt is produced, we can use the pretrained LM to generate text of desired topics efficiently, with the same time cost as standard non-controlled decoding. In comparison, the dedicated steered decoding is often orders-of-magnitude slower, as shown in Table~\ref{table:prompt-generation-speed}.






\section{Related Work}

Standard RL algorithms
can sometimes be over-sensitive to the randomness in the environment. Recent works have considered maximum-entropy RL extensions, such as the soft $Q$-learning (SQL)~\citep{haarnoja2017reinforcement,nachum2017bridging,schulman2017equivalence}, that maximize the entropy of policy besides the rewards, and demonstrated substantial improvement in  robotic and game control~\citep{ziebart2008maximum,ODonoghue2017CombiningPG,nachum2018trustpcl,eysenbach2021maximum}.
Our work is the first to adapt SQL and its advanced variants (in particular the path consistency learning~\citep{nachum2017bridging}) to the challenging text generation problem and show significant results on diverse applications.

Applying RL for text generation has been discussed in alleviating the exposure bias problem and optimizing task metrics~\citep{guo2015generating,li2016deep,wu2016google,rennie2017self,paulus2018a,chen2018fast,liu2020data,pang2021amortized}. For example,~\citet{ranzato2015sequence} used the REINFORCE algorithm~\citep{williams1992simple}, and~\citet{bahdanau2016actor} used the actor-critic algorithm; \citet{guo2018long} and~\citet{shi2018toward} tried to relieve the sparsity problem via hierarchical and inverse RL methods, resp. They are all on-policy RL algorithms with the need of pretraining their models using MLE. RAML~\cite{norouzi2016reward} implicitly relies on the quality of off-policy data; this does not necessarily apply in our experiments with limited good data.%
~\citet{tan2018connecting} and~\citet{hu2022standard} offer a unified view of RAML, RL, and other training methods.
Another line of work focused mostly on using only off-policy data, often for offline training of chatbots~\citep{kandasamy2016batch,zhou2017end,jaques2020human,pang2021text}. As a result, the opportunity of directly improving the reward (as in on-policy updates) for other rich tasks is missed. Our proposed framework combines on- and off-policy training, and further offers solutions for efficient training from scratch in the presence of large action space and sparse sequence-level reward in text generation.


\section{Conclusion}

We present Adaptively Calibrated Critics (ACC), a general off-policy algorithm that learns a Q-value function with bias calibrated TD targets. 
The bias correction in the targets is determined via a parameter that is adjusted by comparing the current value estimates with the most recently observed on-policy returns.
Our method incorporates information from the unbiased sample returns into the TD targets while keeping the high variance of the samples out. We apply ACC to TQC, a recent off-policy continuous control algorithm that allows fine-grained control of the TD target scale through a hyperparameter tuned per environment.
With ACC, this parameter can automatically be adjusted during training,  obviating the need for extensive tuning.
The strong experimental results suggest that our method provides an efficient and general way to control the bias occurring in TD learning. 

Interesting directions for future research are to evaluate the effectiveness of ACC applied to algorithms that work with discrete action spaces and when learning on a real robot where tuning of hyperparameters is very costly. 











\section*{Limitations}



A well-documented limitation of RL methods is the importance of the reward function. The proposed methods are no different in this aspect. This is especially relevant as our reward function could involve a learned model itself, which we proactively leveraged in Sec.~\ref{subsec:adversarial-attack}. We refer interested readers to~\citet{deng2022rlprompt} for more algorithmic considerations.
We also noticed that adapting the pretraining-finetuning paradigm to the proposed methods requires careful designs. A hypothesis points to the discrepancy between MLE objectives (commonly used in pretraining context) and SQL objectives. As discussed in Sec.~\ref{sec:sql-formulation}, the SQL formulation re-interprets the ``logit'' as the $Q$-value, for many good reasons. However, our preliminary experiments suggest that, as a downside, this makes finetuning an MLE-trained model with SQL objectives more challenging.
Future work to scale the proposed methods to tasks such as machine translation and language modeling, and with significantly larger and (MLE-)pretrained models would be exciting.


\section*{Ethics Statement}
This work develops a new RL formulation for text generation. While we demonstrate the framework in four applications, it could be adapted to other (emerging) applications. One major component in these applications is the design of the reward function, which influences the behavior of the trained agent. While we believe the MaxEnt RL framework is more robust against reward misspecification~\citep{eysenbach2021maximum}, the potential failures of sub-optimal reward functions are widely known and discussed.\footnote{\url{https://openai.com/blog/faulty-reward-functions/}} To this end, deploying this model to the wild requires careful and extensive examination, using tools such as~\citet{ribeiro2020beyond}. Further, we highlight the application for (black-box) adversarial attacks in the paper, with the intention of using adversarial attacks to understand the model's inner workings. That being said, this could potentially be misused to conduct malicious attacks against systems. Hence, users of this framework might want to conduct adversarial attacks against their own models to avoid being attacked by other people with bad intentions.

\section*{Acknowledgement}
We thank all reviewers for their invaluable comments and feedback. This research was supported by NSF IIS1563887, NSF CCF1629559, NSF IIS1617583, NGA HM04762010002, NSF IIS1955532, NSF CNS2008248, NSF IIS2123952, and NSF BCS2040381. The views in this article are those of the authors and not the funding agency.





































\bibliography{citations}

\clearpage

\appendix
\setcounter{table}{0}
\setcounter{figure}{0}
\setcounter{algorithm}{0}
\renewcommand{\thetable}{A.\arabic{table}}
\renewcommand{\thefigure}{A.\arabic{figure}}
\renewcommand{\thealgorithm}{A.\arabic{algorithm}} 
% \onecolumn
\section{Proof of Theorem 1}

% The estimator  $\qhatpi_{\beta^*}\sa$ was defined via
% \begin{equation}
%     \beta^* \sa = \argmin_{\beta \in [\bmin, \bmin]} \Bigg| \qhat_\beta \sa - \frac{1}{N} \sum_{i=1}^{N} R_i \sa  \Bigg| .
% \end{equation}

% To declutter the notation we drop the dependencies on the state-action pairs $\sa$ and the policy $\pi$.
Let $\Bar{R} = \frac{1}{N} \sum_{i=1}^{N} R_i $.
First note that the average of symmetrically distributed random variables is still a symmetric distributed random variable and hence 
$\Bar{R} $ is symmetrically distributed.
By assumption $\qhat_{\bmin}$ and $\qhat_{\bmax}$ have the same  distance to the true Q-value which is the mean $Q=\E[\Bar{R}] $, i.e. there is a real value $d$ such that
$Q = \qhat_{\bmin} +d = \qhat_{\bmax} -d$
denote the tail probability  P($\Bar{R} < \qhat_{\bmin}) = p_t$.  
Because of the symmetry and the same distance to the mean we also have that  $P(\Bar{R} > \qhat_{\bmax}) = p_t$.
% In the computation of $\E [ \qhat_{\beta^*} ]$  we can differentiate three events.
% If $\qhat_{\bmin} \leq \Bar{R} \leq \qhat_{\bmax}$ then $\qhat_{\beta^*} =\Bar{R}$, 
% if $\qhat_{\bmin} \geq \Bar{R}$ then $\qhat_{\beta^*} =\qhat_{\bmin}$
% and if $\qhat_{\bmin} \geq \Bar{R}$ then $\qhat_{\beta^*} =\qhat_{\bmax}$.
% We denote the indicator function with $\idc{A}$, which is equal to $1$ if the event $A$ is true and $0$ otherwise.
Then we get
\vspace{-0.1cm}
\begin{align*}
   &\E \big[ \qhat_{\beta^*} \big] 
   = \E \big[ \idc{\qhat_{\bmin} \leq \Bar{R} \leq \qhat_{\bmax}} \Bar{R}  \big]  \\
  & ~+  \E \big[ \idc{\qhat_{\bmin} \geq \Bar{R} }  \qhat_{\bmin} \big] +  \E \big[ \idc{\qhat_{\bmax} \leq \Bar{R} }  \qhat_{\bmax} \big]  \\
    &= (1-2p_t) \cdot \E[\Bar{R}] + p_t \E\big[\qhat_{\bmin}\big] + p_t \E\big[\qhat_{\bmax}\big] \\
   &= (1-2p_t) Q + p_t \qhat_{\bmin}+ p_t \qhat_{\bmax}\\
   &= (1-2p_t) Q + p_t (Q-d) + p_t (Q+d) = Q.
%   &= (1-2p_t) Q + 2p_t Q + p_t(d-d) \\
%   & = Q
\end{align*}




%  long proof

% The estimator  $\qhatpi_{\beta^*}\sa$ was defined via
% \begin{equation}
%     \beta^* \sa = \argmin_{\beta \in [\bmin, \bmin]} \Bigg| \qhat_\beta \sa - \frac{1}{N} \sum_{i=1}^{N} R_i \sa  \Bigg| .
% \end{equation}

% To declutter the notation we drop the dependencies on the state-action pairs $\sa$ and the policy $\pi$.
% Further we write $\Bar{R} = \frac{1}{N} \sum_{i=1}^{N} R_i $.
% First note that the average of symmetrically distributed random variables is still a symmetric distributed random variable and hence 
% $\Bar{R} $ is symmetrically distributed.
% By assumption $\qhat_{\bmin}$ and $\qhat_{\bmax}$ have the same  distance to the true Q-value which is the mean $Q=\E[\Bar{R}] $, i.e. there is a real value $d$ such that
% $Q = \qhat_{\bmin} +d = \qhat_{\bmax} -d$
% denote the tail probability  P($\Bar{R} < \qhat_{\bmin}) = p_t$.  Because of the symmetry and the same distance to the mean we also have that  $P(\Bar{R} > \qhat_{\bmax}) = p_t$.
% In the computation of $\E [ \qhat_{\beta^*} ]$  we can differentiate three events.
% If $\qhat_{\bmin} \leq \Bar{R} \leq \qhat_{\bmax}$ then $\qhat_{\beta^*} =\Bar{R}$, 
% if $\qhat_{\bmin} \geq \Bar{R}$ then $\qhat_{\beta^*} =\qhat_{\bmin}$
% and if $\qhat_{\bmin} \geq \Bar{R}$ then $\qhat_{\beta^*} =\qhat_{\bmax}$.
% We denote the indicator function with $\idc{A}$, which is equal to $1$ if the event $A$ is true and $0$ otherwise.
% Then we get

 
% \begin{align*}
%   \E \Big[ \qhat_{\beta^*} \Big] 
% %   &= \E \Bigg[ \min_{\beta \in [\bmin, \bmin]} \Big| \qhat_\beta  - \Bar{R}   \Big| \Bigg] \\
% %   &=  \E \Bigg[ \idc{\qhat_{\bmin} \leq \Bar{R} \leq \qhat_{\bmax}} 
% %   \min_{\beta \in [\bmin, \bmin]} \Big| \qhat_\beta  - \Bar{R}  \Big| \Bigg]  \\
% %   &~~~~+  \E \Bigg[ \idc{\qhat_{\bmin} \geq \Bar{R} }
% %   \min_{\beta \in [\bmin, \bmin]} \Big| \qhat_\beta  - \Bar{R}  \Big| \Bigg] \\
% %   &~~~~+ \E \Bigg[ \idc{\qhat_{\bmax} \leq \Bar{R} }
% %   \min_{\beta \in [\bmin, \bmin]} \Big| \qhat_\beta  - \Bar{R}  \Big| \Bigg] \\
%   &= \E \Bigg[ \idc{\qhat_{\bmin} \leq \Bar{R} \leq \qhat_{\bmax}} \Bar{R}  \Bigg]  \\
%   &~~~~+  \E \Bigg[ \idc{\qhat_{\bmin} \geq \Bar{R} }  \qhat_{\bmin} \Bigg] \\
%   &~~~~+  \E \Bigg[ \idc{\qhat_{\bmax} \leq \Bar{R} }  \qhat_{\bmax}  \Big| \Bigg]  \\
% %   +  \E \Bigg[ \idc{\qhat_{\bmin} \geq \Bar{R} }  \qhat_{\bmin} \Bigg] 
% %   +  \E \Bigg[ \idc{\qhat_{\bmax} \leq \Bar{R} }  \qhat_{\bmax}  \Big| \Bigg]  \\
%   &= (1-2p_t) \cdot \E[\Bar{R}] + p_t \E\Big[\qhat_{\bmin}\Big] + p_t \E\Big[\qhat_{\bmax}\Big] \\
%   &= (1-2p_t) Q + p_t \qhat_{\bmin}+ p_t \qhat_{\bmax}\\
%   &= (1-2p_t) Q + p_t (Q-d) + p_t (Q+d)\\
%   &= (1-2p_t) Q + 2p_t Q + p_t(d-d) \\
%   &= Q
% \end{align*}











% \section{Using Fewer Critic Networks for Faster Runtime}
% \label{app:2_nets}

% \begin{figure*}[b]
% \centering
% \begin{subfigure}{0.48\textwidth}
%   \includegraphics[width=1\linewidth]{images/ablation/2net_one_fig.pdf}
% \end{subfigure}
% \caption{The mean $\pm$ standard deviation over $10$ trials. 
% Results with different choices for the number of critic networks for each algorithm. }
% \label{fig:num_critic_nets}
% \end{figure*}
% Using $5$ critic networks - the default in TQC - to approximate the value function leads to a high runtime of the algorithm. It is possible to trade off performance against runtime by changing the number of critic networks. We evaluated ACC applied to TQC with $2$ networks and compare it to the standard setting with $5$ networks in Figure \ref{fig:num_critic_nets}. The results show that reducing the number of critic networks to $2$ leads only to a small drop in performance while the runtime is more than $2$ times faster.


% \section{Pseudocode}
% \label{app:pseudocode}


% \begin{algorithm}[t]
%   \caption{ACC - General}
%   \label{alg:general_acc}
% \begin{algorithmic}
%   \STATE {\bfseries Initialize:} bias controlling parameter $\beta$, steps between $\beta$ updates $T_\beta$, $t_\beta = 0$
%   \FOR{$t=1$ {\bfseries to} total number of environment steps}
%   \STATE Interact with environment according to $\pi$, store transitions in replay buffer $\mathcal{B}$ and store observed returns $R\sa$, increment $t_\beta \pluseq 1$
%   \IF{episode ended \textbf{and} $t_\beta >= T_\beta$}
%   \STATE Update $\beta$ with Eq. \ref{eq:one_step_beta_update} using the most recent experience and set $t_\beta=0$
%   \ENDIF
% %   \FOR{$j=1$ {\bfseries to} number Q-updates per environment step}
%   \STATE Sample mini-batch $b$ from $\mathcal{B}$
%   \STATE Update $Q$ with target computed from $\qbeta$ and $b$
% %   \ENDFOR
%   \ENDFOR
% \end{algorithmic}
% \end{algorithm}


% In Algorithm \ref{alg:general_acc} the general version of ACC is presented.
% The pseudocode for ACC applied to TQC is in Algorithm \ref{alg:acc_applied_to_tqc}.
% As the number of dropped targets per network is given by $d= d_{\max} - \beta $, we state the pseudocode in terms of the parameter $d$ instead of $\beta$. 


% \begin{algorithm}[t]
%   \caption{ACC - Applied to TQC}
%   \label{alg:acc_applied_to_tqc}
% \begin{algorithmic}
%   \STATE {\bfseries Initialize:} $d$ the bias controlling parameter, $\alpha$ the learning rate for $d$, $T_d$ the minimum number of steps between updates to $d$, $T_d^{init}$  the initial steps before $d$ is updated,
%   $S_R$ the size from which on episodes are removed from the batch storing the most recent returns, moving average parameter $\tau_d$, $t_d = 0$
%   \FOR{$t=1$ {\bfseries to} total number of environment steps}
%   \STATE Interact with environment according to $\pi$, store transitions in replay buffer $\mathcal{B}$ and, increment $t_d \pluseq 1$
%     \IF{episode ended}
%   \STATE Store observed returns $R\sa$ and corresponding state-action pairs $\sa$ in $\mathcal{B}_R$

   
%   \IF{ $t_d >= T_d$ \textbf{and} $t > T_d^{init}$}
%   \STATE $C = \sum_{ (s,a, R) \in \mathcal{B}_R} \Big[   Q \sa - R \sa  \Big]$,     $ma = (1-\tau_d) ma + \tau_d C$
%   \STATE $d = d + \alpha \frac{C}{ma}$, clip $d$ in interval $[0, d_{max}]$, set $t_d=0$
%     \STATE Remove the oldest episodes from $\mathcal{B}_R$ until there are at most $S_R$ left
%   \ENDIF
%   \ENDIF
% %   \FOR{$j=1$ {\bfseries to} number Q-updates per environment step}
%   \STATE Sample mini-batch from $\mathcal{B}$
%   \STATE Update critic $Q$ as in TQC, where $dN$ (rounded to the next integer) number of targets are dropped from the set of pooled targets 
%   \STATE Update policy $\pi$ as in TQC 
% %   \ENDFOR
%   \ENDFOR
% \end{algorithmic}
% \end{algorithm}


 








% \section{Hyperparameters}
% \label{app:hyperparameter}

% At the beginning of the training we initialize $\beta = 2.5$ and set the step size parameter to $\alpha=0.1$.
% After $T_\beta = 1000$ steps since the last update and when the next episode finishes, $\beta$ is updated with a batch that stores the most recent state-action pairs encountered in the environment and their corresponding observed discounted returns. 
% The choice of $T_\beta$ was motivated by the fact that the maximum duration of an episode is $1000$ steps for the considered environments.
% After every update of $\beta$ the oldest episodes in this stored batch are removed until there are no more than $5000$ state-action pairs left. This means that on average $\beta$ is updated with a batch whose size is a bit over $5000$. 
% The updates of $\beta$ are started as soon as $25000$ environment steps as completed and
% the moving average parameter in the normalization of the $\beta-$update is set to $0.05$. 
% The  first $5000$ environment interactions are generated with a random policy after which learning starts.
% Apart from that all hyperparameters are the same as in TQC with $N=5$ critic networks.
% In Table \ref{tab:hyperparameter} we list all hyperparameters of ACC applied to TQC.

% In the following we also desribe the process of hyperparameter selection.
% The range of values $d$ is allowed to take is set to the interval $[0,5]$ as it includes the optimal hyperparameters for TQC from all environments, which are in the set $\{0,2,5\}$. We did not try higher values than $5$.
% The initial value for number of dropped targets per network was set to $2.5$ as this value is in the middle of the allowed range and did not evaluated other choices.
% The learning rate $\alpha$ of $d$ was set to $0.1$ based on visual inspection of how fast $d$ changes. We  evaluated $\alpha=0.05$ for a small subset of tasks and seeds, but $\alpha=0.1$ gave slightly better results.
% $T_d$ was set to $1000$ as the episode length is $1000$ and we did not evaluate other choices.
% For $T_d^{init}$ we evaluated the choices $10000$ and $25000$ on a small subset of environments and seeds and did not found a big impact on performance. As $d$ changes very quickly in the beginning we chose $T_d^{init}=25000$.
% For $S_R$ we evaluated the choices $1000$ and $5000$ also on a small subset of environments and seeds and found $5000$ to perform slightly better.
% We did not tune the moving average parameter and set it to $\tau_d = 0.05$.
% For all hyperparameters for which we evaluated more than one choice we do not have definite results as the number of seeds and environments were limited.
% The hyperparameters shared with TQC were not changed.




% \begin{table}[h]
% \caption{Hyperparameters values.}
% \label{tab:hyperparameter}
% \vskip 0.15in
% \begin{center}
% \begin{small}
% \begin{sc}
% \begin{tabular}{lccc}
% \toprule
% Hyperparameter & \multicolumn{3}{c}{ACC} \\
% \midrule
% Optimizer & \multicolumn{3}{c}{Adam} \\
% Learning rate & \multicolumn{3}{c}{\num{3e-4}} \\
% Discount $\gamma$ & \multicolumn{3}{c}{0.99} \\
% Replay buffer size & \multicolumn{3}{c}{\num{1e6}} \\
% Number of critics $N$ & \multicolumn{3}{c}{5} \\
% Number of atoms $M$ & \multicolumn{3}{c}{25}\\
% Huber loss parameter  & \multicolumn{3}{c}{1} \\
% Number of hidden layers in critic networks & \multicolumn{3}{c}{3}\\
% Size of hidden layers in critic networks & \multicolumn{3}{c}{512} \\
% Number of hidden layers in policy network & \multicolumn{3}{c}{2} \\
% Size of hidden layers in policy network & \multicolumn{3}{c}{256} \\
% Minibatch size & \multicolumn{3}{c}{256} \\
% Entropy target  &  \multicolumn{3}{c}{$- \dim \mathcal{A}$} \\
% Nonlinearity & \multicolumn{3}{c}{ReLU} \\
% Target smoothing coefficient  & \multicolumn{3}{c}{0.005} \\
% Target updates per critic gradient step & \multicolumn{3}{c}{1} \\
% Critic gradient steps per iteration & \multicolumn{3}{c}{1} \\
% Actor gradient steps per iteration & \multicolumn{3}{c}{1} \\
% Environment steps per iteration & \multicolumn{3}{c}{1} \\
% \midrule
% Initial value for number of dropped targets per network & \multicolumn{3}{c}{2.5} \\
% Maximum value for $d$ denoted $d_\max$ & \multicolumn{3}{c}{5} \\
% Minimum value for $d$ denoted $d_\min$ & \multicolumn{3}{c}{0} \\
% Learning rate for $d$ denoted $\alpha$ & \multicolumn{3}{c}{0.1} \\
% Minimum number of steps between updates to $d$ denoted $T_d$ & \multicolumn{3}{c}{1000} \\
% Initial number of steps before $d$ is updated denoted $T_d^{init}$ & \multicolumn{3}{c}{25000} \\
% Limiting size for batch used to update $d$ denoted $S_R$ & \multicolumn{3}{c}{5000} \\
% Moving average parameter $\tau_d$ & \multicolumn{3}{c}{0.05} \\
% \midrule
% \midrule
% Hyperparameter in Sample Efficient Experiment & ACC\_1q & ACC\_2q & ACC\_4q \\
% \midrule
% Critic gradient steps per iteration & 1 & 2 & 4 \\
% Actor gradient steps per iteration & 1 & 1 & 1 \\
% Target updates per critic gradient step & 1 & 1 & 1\\
% \bottomrule
% \end{tabular}
% \end{sc}
% \end{small}
% \end{center}
% \vskip -0.1in
% \end{table}




% $d$ the bias controlling parameter, $T_d$ the number of steps between updates to $d$, $T_d^{init}$  the initial steps before $d$ is updated,
%   $S_R$ the size from which on episodes are removed from the batch storing the most recent returns, moving average parameter $\tau_d$, $t_d = 0$


% \section{Potential Limitations}
% One limitation of our work is that ACC can not be applied in the offline RL setting, as ACC also uses on-policy data.
% Furthermore, in the stated form ACC relies on the episodic RL setting. However, we believe that ACC could potentially be adapted to that setting. 
% It is also not entirely clear how the algorithm would perform in the terminal reward setting, where a reward of for example $1$ is given upon successful completion of a specific task. While we do not have experiments for such environments we imagine that the positive effect of ACC could diminish as the true Q-values of states closer to the start of the episode are almost zero because of the discounting. 



% \section{Potential Negative Societal Impact}
% This work studied general and fundamental algorithms in the area of deep reinforcement learning. 
% Therefore, we do not see how this work could lead to a negative societal impact.



% \section{Analysis of the ACC Parameter}
% \label{app:analysis_acc_parameter}

% \begin{figure*}[b]
% \begin{subfigure}[b]{0.98\textwidth}
%   \includegraphics[width=1\linewidth]{images/analysis/visualize_beta_all_envs.pdf}
% \end{subfigure}
% \begin{subfigure}[b]{0.98\textwidth}
%   \includegraphics[width=1\linewidth]{images/analysis/visualize_beta_one_run_all_envs.pdf}
% \end{subfigure}

% \caption{Development of the number of dropped targets per network $d=d_{max}-\beta$ in ACC over time for different environments. The top row shows the mean (thick line) and standard deviation (shaded area) over the $10$ trials where for readability a uniform filter of size $15$ is used.
% The bottom row shows the unfiltered development for one of the seeds.}
% \label{fig:num_dropped_targets_all_envs}
% \end{figure*}

% % \begin{figure*}
% %     \centering
% %     \includegraphics[width=0.98\textwidth]{images/analysis/visualize_beta_all_envs.pdf}
% %     \includegraphics[width=0.98\textwidth]{images/analysis/visualize_beta_one_run_all_envs.pdf}
% %     \caption{Caption}
% %     \label{fig:my_label}
% % \end{figure*}



% To better understand the hidden training dynamics of ACC we show in 
% Figure \ref{fig:num_dropped_targets_all_envs}
% how the number of dropped targets per network $d=d_{max}-\beta$ evolves during training. 
% To do so we plotted $d$ after every $5000$ steps during the training of ACC.
% From the top row the first observation is that per environment the results are similar over the $10$ seeds as can be seen from the relatively low standard deviation. We show the single runs for all seeds in the appendix to further support this observation.
% However, there are large differences between the environments which supports the argument that it might not be possible to find a single hyperparameter that works well on a wide variety of different environments.
% Another point that becomes clear from the plots is that the optimal amount of overestimation correction might change over time during the training even on a single environment.

% In the bottom row of Figure \ref{fig:num_dropped_targets_all_envs} we plotted the evolution of $d$ for one of the $10$ trials in order to shed light on the actual training mechanics of a single run without lost information due to averaging.
% For each environment there is a trend but $d$ is also fluctuating to a certain degree.
% While this shows that the initial value of $d$ is not very important as the value quickly changes, this also highlights another interesting aspect of ACC. 
% The rollouts give highly fluctuating returns. The parameter $d=d_{max}-\beta$ is changing more slowly and picks up the trend. So a lot of the variance of the returns is filtered out in ACC by incorporating on-policy samples via the detour over $\beta$.
% This leads to relatively stable TD targets computed from $\qbeta$ while an upbuilding under- or overestimation is prevented as $\beta$ picks up the trend. On the other hand, if $\beta$ would change too slowly the upbuilding of the bias might not be stopped.







\end{document}
