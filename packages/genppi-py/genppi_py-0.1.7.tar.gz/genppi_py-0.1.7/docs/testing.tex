% --- START OF FILE testing_en.tex ---

\documentclass[11pt, a4paper]{article}

% BASIC PACKAGES
\usepackage[utf8]{inputenc}
\usepackage[T1]{fontenc}
\usepackage[english]{babel} % Changed to English
\usepackage{geometry}
\geometry{a4paper, margin=1in}

% FONTS AND TITLES
\usepackage{tgpagella} % Nice text font
\usepackage{titlesec}
\titleformat{\section}{\normalfont\Large\bfseries}{\thesection}{1em}{}
\titleformat{\subsection}{\normalfont\large\bfseries}{\thesubsection}{1em}{}
\titlespacing*{\section}{0pt}{3.5ex plus 1ex minus .2ex}{2.3ex plus .2ex}

% CODE FORMATTING
\usepackage{xcolor}
\usepackage{listings}

\definecolor{codegray}{rgb}{0.5,0.5,0.5}
\definecolor{codepurple}{rgb}{0.58,0,0.82}
\definecolor{codebackground}{rgb}{0.95,0.95,0.95}

\lstdefinestyle{bashstyle}{
    backgroundcolor=\color{codebackground},   
    commentstyle=\color{codegray},
    keywordstyle=\color{blue},
    stringstyle=\color{codepurple},
    basicstyle=\ttfamily\small,
    breakatwhitespace=false,         
    breaklines=true,                 
    captionpos=b,                    
    keepspaces=true,                 
    numbers=left,                    
    numbersep=5pt,                   
    showspaces=false,                
    showstringspaces=false,
    showtabs=false,                  
    tabsize=2,
    frame=single,
    rulecolor=\color{black!20}
}

% OTHERS
\usepackage{hyperref}
\hypersetup{
    colorlinks=true,
    linkcolor=blue,
    filecolor=magenta,      
    urlcolor=cyan,
    pdftitle={Testing Guide for GenPPI}, % Translated
    pdfpagemode=FullScreen,
}

% --- DOCUMENT START ---

\title{\bfseries Testing Guide for the GenPPI Python Interface}
\author{}
\date{}

\begin{document}
\maketitle

This document provides a simplified guide to installing and testing the GenPPI Python interface. It focuses on the most common use cases.

\section{Environment Setup}
Follow these steps to set up your test environment.

\subsection{Installation (Development Mode)}
To test and modify the code, install the package in development mode. This allows your source code changes to be reflected immediately without needing to reinstall.

\begin{lstlisting}[style=bashstyle]
# Navigate to the project's root directory (where setup.py is)
cd /path/to/genppi_py

# Install in development mode
pip install -e .
\end{lstlisting}

\subsection{Installation (Production Version)}
To install the stable version from TestPyPI:

\begin{lstlisting}[style=bashstyle]
# Install version 0.1.2 from TestPyPI (with dependencies from official PyPI)
pip install --index-url https://test.pypi.org/simple/ --extra-index-url https://pypi.org/simple/ genppi-py==0.1.2

# Verify critical dependencies
python -c "import py7zr; print('py7zr:', py7zr.__version__)"
python -c "import multivolumefile; print('multivolumefile OK')"
\end{lstlisting}

\textbf{Important Note:} Version 0.1.2 exclusively uses the py7zr library to extract multi-volume \texttt{model.7z} files. The dependencies \texttt{py7zr>=0.20.0} and \texttt{multivolumefile>=0.2.3} are installed automatically.

\textbf{Note on TestPyPI:} Since TestPyPI does not host all dependencies, you must use the \texttt{--extra-index-url} flag. This allows pip to fetch dependencies (like py7zr and multivolumefile) from the official PyPI.

\subsection{Downloading Test Data}
The package includes a command to easily download the required sample files for testing.
\begin{lstlisting}[style=bashstyle]
# This command will create a 'samples' folder in your current directory
genppi-download-samples
\end{lstlisting}
This will download the necessary protein files and save them in the \texttt{samples/} folder.

\subsection{Verifying the Installation}
After installation, the \texttt{genppi} command should be available in your terminal. Check it by running the help command:
\begin{lstlisting}[style=bashstyle]
genppi --help
\end{lstlisting}
This should display a list of all available parameters, confirming a successful installation.

\textbf{Important:} Use the \texttt{genppi} command, not \texttt{genppi.py}. The Python interface provides a unified command across all operating systems.

\section{Running the Main Tests}
GenPPI is designed to be user-friendly. The following tests cover over 90\% of use cases.

\subsection{Test 1: Standard Execution (Most Common)}
The most common usage is to simply provide the directory with the protein files. \textbf{GenPPI automatically runs conserved neighborhood and phylogenetic profiles (Method 1) when multiple genomes are present.}
\begin{lstlisting}[style=bashstyle]
# Run GenPPI on the sample data (CN + PP Method 1 are automatic)
genppi -dir samples

# Run with gene fusion as well
genppi -dir samples -genefusion
\end{lstlisting}
The program will process the files and generate results in the \texttt{samples/} folder.

\subsection{Test 2: Running with Machine Learning (Higher Accuracy)}
For a more accurate analysis, you can enable the Machine Learning model with the \texttt{-ml} flag. This is the second most common use case.

\textbf{Prerequisite:} The \texttt{model.dat} file must be available. On first run, GenPPI will try to download and extract it automatically. You can also download it manually:
\begin{lstlisting}[style=bashstyle]
genppi-download-model
\end{lstlisting}
\textbf{Running the test:}
\begin{lstlisting}[style=bashstyle]
# Run GenPPI with the Machine Learning model
genppi -dir samples -ml
\end{lstlisting}

\subsection{Test 3: Testing Phylogenetic Profile Methods}
GenPPI offers 7 different methods for phylogenetic profile prediction. Method 1 is the default, but you can choose others using \texttt{-ppmethod}:
\begin{lstlisting}[style=bashstyle]
# Method 1: No filters (default - these commands are equivalent)
genppi -dir samples
genppi -dir samples -ppmethod 1

# Method 2: Only conserved neighborhood interactions
genppi -dir samples -ppmethod 2

# Method 5: Filter by threshold (requires additional parameters)
genppi -dir samples -ppmethod 5 -threshold 5 -plusminus ">"
\end{lstlisting}

\subsection{Test 4: Running with Low Memory Usage (\texttt{--use-db})}
If you are working with many genomes or on a machine with limited memory, use the \texttt{--use-db} flag. This makes GenPPI save intermediate data to disk instead of keeping it in memory.
\begin{lstlisting}[style=bashstyle]
# Combine with standard execution or with -ml
genppi -dir samples -ml --use-db
\end{lstlisting}

\section{Verifying the Results}
After running any of the tests above, check that the result files were created correctly.
\begin{lstlisting}[style=bashstyle]
# List the report and network files
ls -l samples/ppi-report/
ls -l samples/ppi-files/
\end{lstlisting}
You should see a \texttt{report.txt} file and several network files (like \texttt{.sif} and \texttt{.dot}) inside these folders.

\section{Testing Automatic Recovery}
An important feature of the package is its ability to download missing components.

\subsection{Testing the Executable Download}
\begin{enumerate}
    \item \textbf{Remove the executable:}
    \begin{lstlisting}[style=bashstyle]
# (Adjust the path if your structure is different)
rm -f genppi_py/genppi_py/bin/genppi32g-Linux
    \end{lstlisting}
    \item \textbf{Run a simple command:}
    \begin{lstlisting}[style=bashstyle]
genppi --help
    \end{lstlisting}
\end{enumerate}
The package should silently download the missing executable and then display the help message.

\subsection{Testing the ML Model Download}
\begin{enumerate}
    \item \textbf{Remove the model file:}
    \begin{lstlisting}[style=bashstyle]
# (Adjust the path if your structure is different)
rm -f genppi_py/genppi_py/bin/model.dat
    \end{lstlisting}
    \item \textbf{Run a command that requires the model:}
    \begin{lstlisting}[style=bashstyle]
genppi -dir samples -ml
    \end{lstlisting}
\end{enumerate}
You will see messages indicating that \texttt{model.dat} was not found and that the download and extraction will begin.

\section{Common Troubleshooting}
If something goes wrong, check the following points:
\begin{itemize}
    \item \textbf{Permission Issues (Linux/macOS):} If you get a "Permission denied" error, the executables may lack execution permissions. Fix it with:
    \begin{lstlisting}[style=bashstyle]
# For Python installation:
chmod +x genppi_py/genppi_py/bin/genppi*

# For manual download (executables are renamed):
chmod +x genppi genppidb
    \end{lstlisting}
    
    \item \textbf{Failed to Extract \texttt{model.dat}:} Version 0.1.7 uses the py7zr library for multi-volume extraction. If automatic extraction fails:
    \begin{enumerate}
        \item \textbf{Check py7zr installation:}
        \begin{lstlisting}[style=bashstyle]
# Check if py7zr is installed correctly
python -c "import py7zr; print('py7zr version:', py7zr.__version__)"

# If needed, reinstall a specific version
pip install --upgrade --force-reinstall "py7zr>=0.20.0,<1.0.0"
        \end{lstlisting}
        
        \item \textbf{Manual extraction:} If automatic extraction still fails, extract it manually:
        \begin{lstlisting}[style=bashstyle]
# Download the 7z file parts manually
wget https://github.com/santosardr/genppi/raw/master/src/model.7z.001
wget https://github.com/santosardr/genppi/raw/master/src/model.7z.002
wget https://github.com/santosardr/genppi/raw/master/src/model.7z.003

# Extract using py7zr
python -c "
import py7zr
from multivolumefile import MultiVolume
with MultiVolume('model.7z', mode='rb') as vol:
    with py7zr.SevenZipFile(vol, mode='r') as archive:
        archive.extractall(path='.')
"
        \end{lstlisting}
        
        \item After manual extraction, move the \texttt{model.dat} file to the package's bin directory.
    \end{enumerate}
\end{itemize}

With this guide, you can quickly and efficiently validate the core functionalities of your GenPPI package.

\end{document}

% --- END OF FILE testing_en.tex ---