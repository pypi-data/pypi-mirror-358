\documentclass[11pt, oneside]{article}
\usepackage{geometry}
\geometry{letterpaper}
\usepackage{graphicx}
\usepackage[titletoc,toc,title]{appendix}
\usepackage{amssymb}
\usepackage{physics}
\usepackage{array}
\usepackage{makecell}

\usepackage{hyperref}
\hypersetup{
    colorlinks = true
}

\usepackage{cleveref}

\title{Memo: SkyH5 file format}
\author{Bryna Hazelton, and the pyradiosky team}
\date{October 5, 2023}

\begin{document}
\maketitle
\tableofcontents
\section{Introduction}
\label{sec:intro}

This memo introduces a new HDF5\footnote{\url{https://www.hdfgroup.org/}} based
file format of a \texttt{SkyModel} object in
\texttt{pyradiosky}\footnote{\url{https://github.com/RadioAstronomySoftwareGroup/pyradiosky}},
a python package that provides objects and interfaces for representing diffuse,
extended and compact astrophysical radio sources. Here, we describe the required
and optional elements and the structure of this file format, called \textit{SkyH5}.

We assume that the user has a working knowledge of HDF5 and the associated
python bindings in the package \texttt{h5py}\footnote{\url{https://www.h5py.org/}}, as
well as \texttt{SkyModel} objects in \texttt{pyradiosky}. For more information about
HDF5, please visit \url{https://portal.hdfgroup.org/display/HDF5/HDF5}. For more
information about the parameters present in a \texttt{SkyModel} object, please visit
\url{https://pyradiosky.readthedocs.io/en/latest/skymodel.html}.
Examples of how to interact with \texttt{SkyModel} objects in \texttt{pyradiosky} are
available at \url{http://pyradiosky.readthedocs.io/en/latest/tutorial.html}.

Note that throughout the documentation, we assume a row-major convention (i.e.,
C-ordering) for the dimension specification of multi-dimensional arrays. For
example, for a two-dimensional array with shape ($N$, $M$), the $M$-dimension is
varying fastest, and is contiguous in memory. This convention is the same as
Python and the underlying C-based HDF5 library. Users of languages with the
opposite column-major convention (i.e., Fortran-ordering, seen also in MATLAB
and Julia) must transpose these axes.

\section{Overview}
\label{sec:overview}
A SkyH5 object contains data representing catalogs and maps of
astrophysical radio sources, including the associated metadata necessary to interpret
them. A SkyH5 file contains two primary HDF5 groups: the \texttt{Header} group,
which contains the metadata, and the \texttt{Data} group, which contains the Stokes
parameters representing the flux densities or brightness temperatures of the sources
(as well as some optional arrays the same size as the stokes parameters data).
Datasets in the \texttt{Data} group
are can be passed through HDF5's compression
pipeline, to reduce the amount of on-disk space required to store the data.
However, because HDF5 is aware of any compression applied to a dataset, there is
little that the user has to explicitly do when reading data. For users
interested in creating new files, the use of compression is optional in the
SkyH5 format, because the HDF5 file is self-documenting in this regard.

Many of the datasets in SkyH5 files have units associated with them (represented
as \texttt{astropy Quantity} objects on the \texttt{SkyModel} object).
The units are stored as attributes on the datasets with the name ``unit''.
Datasets that derive from other \texttt{astropy} objects (e.g. \texttt{astropy Time,
astropy EarthLocation, astropy Latitude, astropy Longitude})
also have an ``object\_type'' attribute indicating the object type.

In the discussion below, we discuss required and optional datasets in the
various groups. We note in parenthesis the corresponding attribute of a \texttt{SkyModel}
object. Note that in nearly all cases, the names are coincident, to make things
as transparent as possible to the user.

\section{Header}
\label{sec:header}
The \texttt{Header} group of the file contains the metadata necessary to interpret
the data. We begin with the required parameters, then continue to optional
ones. Unless otherwise noted, all datasets are scalars (i.e., not arrays). The
precision of the data type is also not specified as part of the format, because
in general the user is free to set it according to the desired use case (and
HDF5 records the precision and endianness when generating datasets). When using
the standard \texttt{h5py}-based implementation in \texttt{pyradiosky}, this typically
results in 32-bit integers and double precision floating point numbers. Each
entry in the list contains \textbf{(1)} the exact name of the dataset in the
HDF5 file, in boldface, \textbf{(2)} the expected datatype of the dataset, in
italics, \textbf{(3)} a brief description of the data, and \textbf{(4)} the name
of the corresponding attribute on a \texttt{SkyModel} object, italicized and in
parentheses at the end of the entry.

Note that string datatypes should be handled with care. See
the Appendix in the UVH5 memo
(\url{https://github.com/RadioAstronomySoftwareGroup/pyuvdata/blob/main/docs/references/uvh5_memo.pdf})
for appropriately defining them for interoperability between different HDF5
implementations.

\subsection{Required Parameters}
\label{sec:req_params}
\begin{itemize}

\item \textbf{component\_type}: \textit{string} The type of components in the sky model.
The options are: ``healpix'' and ``point''. If component\_type is ``healpix'', the components
are the pixels in a HEALPix map in units compatible with K or Jy/sr. If the
component\_type is ``point'', the components are point-like sources, or point like
components of extended sources, in units compatible with Jy or K sr. Some additional
parameters are required depending on the component type. (\textit{component\_type})

\item \textbf{Ncomponents}: \textit{int} The number of components in the sky model. This
can be the number of individual compact sources, or it can include components of
extended sources, or the number of pixels in a map. (\textit{Ncomponents})

\item \textbf{spectral\_type}: \textit{string} This describes the type of spectral model for
the components. The options are:
\begin{enumerate}
	\item \textbf{spectral\_index} The convention for the spectral index is
	$I=I_0 \frac{f}{f_0}^{\alpha}$, where $I_0$ is the stokes parameter at the
	reference\_frequency parameter $f_0$ and $\alpha$ is the spectral\_index
	parameter. Note that the spectral index is assumed to apply in the units of the
	stokes parameter (i.e. there is no additive factor of 2 applied to convert between
	brightness temperature and flux density units). If the spectral model uses a
	spectral index, the reference\_frequency and spectral\_index parameters are
	required.
	\item \textbf{subband} The subband spectral model is used for catalogs with
	multiple flux measurements at different frequencies (i.e. GLEAM
	\url{https://www.mwatelescope.org/science/galactic-science/gleam/}). For subband
	spectral models, the freq\_array and freq\_edge\_array parameters are required
	to give the nominal (usually the central) frequency and the top and bottom of
	each subband respectively.
	\item \textbf{flat} The flat spectral model assumes no spectral flux dependence,
	which can be useful for testing. Since the flux is assumed to be the same at all
	frequencies it does not require any extra parameters to be defined.
	\item \textbf{full} The full spectral model is used for catalogs with flux values at
	multiple frequencies that are not expected to have flux correlations as a function
	of frequency, so cannot not be interpolated to frequencies not included in the
	catalog. This is a good representation for e.g. Epoch of Reionization signal cubes.
	For full  spectral models, the freq\_array parameter is required to give the frequencies.
\end{enumerate}
(\textit{spectral\_type})

\item \textbf{Nfreqs}: \textit{int}
Number of frequencies if spectral\_type is ``full'' or ``subband'', 1 otherwise.
(\textit{Nfreqs})

\item \textbf{history}: \textit{string} The history of the catalog. (\textit{history})
\end{itemize}


\subsection{Optional Parameters}
\label{sec:opt_params}
\begin{itemize}
\item \textbf{name}: \textit{string} The name for each component. This is a
one-dimensional array of size (Ncomponents).
Note this is \textbf{required} if the component\_type is ``point''. (\textit{name})

\item \textbf{skycoord}:
A nested dataset that contains the information to create an \texttt{astropy SkyCoord}
object representing the component positions. Note this is \textbf{required} if the
component\_type is ``point''. The keys must include:
	\begin{itemize}
	\item \textbf{frame}: \textit{string} The name of the coordinate frame
	(e.g. ``icrs'', ``fk5'', ``galactic''). Must be a frame supported by \texttt{astropy}.
	Note that only one frame is allowed, which applies to all the components.
	\item \textbf{representation\_type}: \textit{string} The representation type, one
	of ``spherical'', ``cartesian'' or ``cylindrical''. This sets what the coordinate
	names can be. It is most common to set this to ``spherical" and
	specify latitudinal and longitudinal coordinates (e.g. \textbf{ra}, \textbf{dec})
	and optionally distance coordinates. It is also possible to use other
	representations such as cartesian or cylindrical, e.g. for ``cartesian" the
	coordinates would be specified in x, y, and z. See the \texttt{astropy SkyCoord}
	docs for more details.
	\item Coordinate names (e.g. \textbf{ra}, \textbf{dec}, \textbf{alt}, \textbf{az}):
	\textit{float} Two or three such components must be present, which ones are
	required depend on the frame and representation\_type. These are
	one-dimensional arrays of size (Ncomponents).
	\end{itemize}
And may include any other attributes accepted as input parameters for an
\texttt{astropy SkyCoord} object (e.g. obstime, equinox, location).
Each of these datasets may have ``unit'' and ``object\_type'' attributes and may
be either a scalar or a one-dimensional array of size (Ncomponents) as
appropriate.
(\textit{skycoord})

\item \textbf{nside}: \textit{int}
The HEALPix nside parameter. Note this is \textbf{required} if the
component\_type is ``healpix'' and should not be defined otherwise.
(\textit{nside})

\item \textbf{hpx\_order}: \textit{string}
The HEALPix pixel ordering convention, either ``ring'' or ``nested''.
Note this is \textbf{required} if the component\_type is ``healpix'' and should
not be defined otherwise. (\textit{hpx\_order})

\item \textbf{hpx\_frame}:  A nested dataset that contains the information to
describe an \texttt{astropy} coordinate frame giving the HEALPix coordinate
frame. This is similar to the skycoord dataset described above but it does not
contain the representation\_type or the coordinate names.
Note this is \textbf{required} if the component\_type is ``healpix'' and should
not be defined otherwise. The keys must include:
	\begin{itemize}
	\item \textbf{frame}: \textit{string} The name of the coordinate frame
	(e.g. ``icrs'', ``fk5'', ``galactic''). Must be a frame supported by
	\texttt{astropy}.
	\end{itemize}
And may include any other scalar attributes accepted as input parameters
for an \texttt{astropy SkyCoord} object (e.g. obstime, equinox, location). Each
of these datasets may have ``unit'' and ``object\_type'' attributes as appropriate.
(\textit{hpx\_frame})

\item \textbf{hpx\_inds}: \textit{int}
The HEALPix indices for the included components. Does not need to include
all the HEALPix pixels in the map. This is a one-dimensional array of size
(Ncomponents). Note this is \textbf{required} if the component\_type is
``healpix'' and should not be defined otherwise. (\textit{hpx\_inds})

\item \textbf{freq\_array}: \textit{float}
Frequency array giving the nominal (or central) frequency in a unit that can be
converted to Hz. Note this is \textbf{required} if the spectral\_type is ``full'' or
``subband'' and should not be defined otherwise. (\textit{freq\_array})

\item \textbf{freq\_edge\_array}: \textit{float}
Array giving the frequency band edges in a unit that can be converted to Hz.
This is a two dimensional array with shape (2, Nfreqs). The zeroth index in
the first dimension holds the lower band edge and the first index holds the
upper band edge. Note this is \textbf{required} if the spectral\_type is
``subband'' and should not be defined otherwise. (\textit{freq\_edge\_array})

\item \textbf{reference\_frequency}: \textit{float}
Reference frequency giving the frequency at which the flux in the stokes
parameter was measured in a unit that can be converted to Hz. This is a
one-dimensional array of size (Ncomponents). Note this is \textbf{required} if
the spectral\_type is ``spectral\_index'' and should not be defined if the
spectral\_type is ``full'' or ``subband''.
(\textit{reference\_frequency})

\item \textbf{spectral\_index}: \textit{float}
The spectral index describing the flux evolution with frequency, see details
in the spectral\_type description above. This is a one-dimensional array
of size (Ncomponents). Note this is \textbf{required} if the spectral\_type is
``spectral\_index'' and should not be defined otherwise.
(\textit{spectral\_index})

\item \textbf{extended\_model\_group}: \textit{string}
Identifier that groups components of an extended source model.
This is a one-dimensional array of size (Ncomponents), with empty strings
for point sources that are not components of an extended source.
(\textit{extended\_model\_group})
\end{itemize}

\subsection{Extra Columns}
\label{sec:extra-columns}
\texttt{SkyModel} objects support ``extra columns'', which are additional arbitrary
per-component arrays of metadata that are useful to carryaround with the data but
which are not formally supported as a reserved keyword in the \texttt{Header}. In a
SkyH5 file, extra columns are handled by creating a datagroup called
\textbf{extra\_columns} inside the \texttt{Data} datagroup. When possible, these
quantities should be HDF5 datatypes, to support interoperability between SkyH5
readers. Inside of the extra\_columns datagroup, each extra columns is saved as
a key-value pair using a dataset, where the name of the extra columns is the name
of the dataset and its corresponding array is saved in the dataset. The ``unit'' and
``object\_type'' HDF5 attributes are used in the same way as for the other header
items (see \ref{sec:overview}), but it is not recommended to use other attribute
names due to the lack of support inside of \texttt{pyradiosky} for ensuring the
attributes are properly saved when reading and writing SkyH5 files.

\section{Data}
\label{sec:data}
In addition to the \texttt{Data} datagroup in the root namespace, there must be
one called \texttt{Data}. This datagroup saves the Stokes parameters representing
the flux densities or brightness temperatures of the sources and some optional
arrays that are the same size. They are also all expected to be the same shape:
(4, Nfreqs, Ncomponents) where the first dimension indexes the polarization
direction, ordered (I, Q, U, V). The \textbf{stokes} dataset must be present in this
datagroup and it must have a ``unit'' attribute that is equivalent to Jy or K str if the
component\_type is ``point'' or equivalent to Jy/str or K if the component\_type is
``healpix''. In addition, this datagroup may also contain a  \textbf{stokes\_error}
dataset that gives the standard error on the stokes values and should have the
same ``unit'' attribute as the stokes dataset and a  \textbf{beam\_amp} dataset
that gives the beam amplitude at the source position for the instrument that made
the measurement.

\end{document}
